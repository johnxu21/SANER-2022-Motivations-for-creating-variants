\section{Research Questions}
\label{sec:rqs}

Our examination of the literature revealed that while several
studies have investigated variant forks, only one of them has been performed on \gh.
No work has extensively examined the motivations behind creating variant forks in the \gh days.
To this end, the overall goal of this research is to understand the developers' motivation behind creating a variant fork in the context of social coding platforms like \gh. We also want to understand if the variant forks and their upstream repositories continue collaboration during the co-evolution. 
Consequently, our first question explores the developers' motivation behind creating a variant fork.

\nd  \textbf{\rqOne} With this research question, we aim to investigate if the creators of the variant were part core developers of upstream or not, and whether the upstream and variant currently have active maintainers.

\nd  \emph{\rqOneOne}

\nd \emph{\rqOneTwo}

\nd \textbf{\rqTwo} With this research question, we want to investigate if the motivations reported in literature (see Section~\ref{sec:background}) still hold or there are new ones in this \gh era.
To understand the motivations behind the creation and maintenance of variant forks, we ask two questions:

\nd \emph{\rqTwoOne} 

\nd  \emph{\rqTwoTwo} 

\nd  \emph{\rqTwoThree}

\nd \textbf{\rqThree} In Section~\ref{sec:background}, the literature based on quantitative analysis of variants from three ecosystems revealed that there is limited interaction between variants and upstream repositories. With this research question, we want to ascertain if this holds by performing a qualitative study. We asked the following subquestions.

\nd \textit{\rqThreeOne}

\nd \textit{\rqThreeTwo}
