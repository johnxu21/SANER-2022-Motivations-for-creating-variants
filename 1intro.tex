\section{Introduction}
\label{sec:intro}
There is an increasing popularity of maintaining open source projects on social coding platforms like \gh. This has substantially improved both code reuse and collaborative development through forking of software repositories as well as sharing of code changes though pull requests and many \git facilities available.
A developer may fork a \textit{mainline repository}
%\textit{A}
into a new \textit{forked repository}, typically transforming governance over the latter to a new developer, while preserving the full revision history and establishing traceability information. 
Forking was rare and was typically intended to compete with the original project~\cite{Linus:2012Perspectives,Gregorio:2012,Viseur:2012Forks,Linus:2013CodeForking,Linus:2011ToFork,Gamalielsson:2014Sustainability}.
The community typically distinguishes between two kinds of forks~\cite{Zhou:2020}.
\textit{Social forks} that are created for isolated development with the goal of contributing back to the mainline. \textit{Variant forks} that are created for splitting off a new development branch, often to steer the development into another direction without intending to contribute back, while leveraging the mainline project that defines or adheres to some standards~\cite{sung:ICSE:2020}.

A lot of studies have already investigated variant forking of free and open-source projects focusing on identifying the motivations behind variant forks~\cite{Linus:2012Perspectives,Gregorio:2012,Viseur:2012Forks,Linus:2013CodeForking,Linus:2011ToFork,Gamalielsson:2014Sustainability}. However, most of these studies has been conducted before the rise of social coding, much of it on \texttt{SourceForge}, before the popularity of social coding platforms like \gh. 
We know of only two studies that have so far investigated reasons for forking~\cite{businge:2018icsme,Zhou:2020}. Zhou et al.~\cite{Zhou:2020} report that perceptions and practices of variant forking in the \gh era changed significantly. Variant forks often evolve out of social forks rather than being planned deliberately and developers frequently perceive them as alternatives to the original projects.

This study builds on the previous studies to further investigate the motivations behind the creation and maintenance of variants on \gh the largest social coding platform. To achieve our goal, we performed an exploratory investigation of maintainers of variants hosted on \gh. We conducted an online survey that was answered by 112 maintainers of active variant projects.

\nd 

