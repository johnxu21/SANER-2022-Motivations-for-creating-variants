% !TEX root = 0main.tex
\section{Introduction}
\label{sec:intro}

The collaborative nature of open source software development has led to the advent of social coding platforms centred around the git version control system, such as \gh, BitBucket, and GitLab.
These platforms bring the collaborative nature and code reuse of open source software development to another level, via facilities like forking, pull requests, cherry-picking, etc.
Developers may fork a \textit{mainline repository} into a new \textit{forked repository} and take governance over the latter while preserving the full revision history of the former.
Before the advent of social coding platforms, forking was rare and was typically intended to compete with the original project~\cite{Linus:2012Perspectives,Gregorio:2012,Viseur:2012Forks,Linus:2013CodeForking,Linus:2011ToFork,Gamalielsson:2014Sustainability}.

With the rise of the pull-based development model~\cite{Gousios:2014ICSE}, forking has become more common and the community typically characterises forks by their purpose~\cite{Zhou:2020}.
\textit{Social forks} are created for isolated development with the goal of contributing back to the mainline.
\textit{Variant forks} are, in contrast, created by splitting off a new development branch
to steer development into a new direction, while leveraging the mainline project that defines or adheres to some standards~\cite{sung:ICSE:2020}.
% without necessarily intending to contribute back to the mainline.\cd{Are these definitions correct? Does a variant fork become a social fork if it is contributed back? If the real distinguishing factor is the next sentence, it should be incorporated in the definition.}

Several studies have investigated the motivations behind variant forks in the context of open source projects~\cite{Linus:2012Perspectives,Gregorio:2012,Viseur:2012Forks,Linus:2013CodeForking,Linus:2011ToFork,Gamalielsson:2014Sustainability}.
However, most have been conducted before the rise of social coding platforms and it is known that \gh has significantly changed the perception and practices of forking~\cite{Zhou:2020}.
%Developers tend to perceive social forks as alternatives to the original mainline project~\cite{Zhou:2020}
%\cd{Is this supported by the literature? \jb{yes, ~\cite{Zhou:2020}}}, and variant forks
In the social coding platforms era, variant projects often evolve out of social forks rather than being planned deliberately~\cite{Zhou:2020}.
To this end, the social coding platforms often enable mainlines and variants projects to peacefully co-exist rather than compete.
%\cd{From what I've read so far, I cannot conclude that variant forks are not competing for the attention of the same developers/end users. Perhaps you need to stress the reason why more. It will be important for the motivation of this work.}, 
Little is know on the motivations of creating variants in the \gh era, in this sense, it is worthwhile
%\cd{understatement} 
to revisit the motivation for creating variant forks (\textit{why?}).

Social coding platforms like \gh offer many facilities for code sharing (i.e., pull requests, cherry-picking, etc).
So if projects co-exist, one would expect that variant forks take advantage of this common ancestry, and frequently exchange interesting updates, for example, patches on the common artifacts.
Despite advanced code-sharing facilities, Businge et al. observed very limited code integration, using the \git and \gh based facilities, between the mainline and the variant projects~\cite{businge:emse:2021}.
%\cd{From the above sentence it is not clear whether the study included ad-hoc reuse, i.e., the sharing of code snippets in the second sentence of the paragraph. It reads as if only the structured tool-supported code sharing facilities were used.  It that interpretation is not correct, you might want to consider rephrasing the description, as you use it as a motivation in the next sentence.}
This suggests that the code sharing facilities in themselves are not enough for graceful co-evolution, so in that sense it is worthwhile to investigate impediments for co-evolution (\textit{how?}).

%\sd{Be careful: I massaged the RQ slightly. If you copy them in the experimental set-up use the commands I created for them.}

\noindent
We will therefore study two research questions:

%\begin{itemize}
%\item
\textit{\textbf{\RQOne}}
The literature pre-dating \git and social coding platforms identified three categories of motivations for creating variant forks: technical (e.g., diverging features), governance (e.g., diverging interests), legal (e.g., diverging licences), and personal (e.g., diverging principles).
This research question aims to investigate whether those motivations for variant forks are still the same or whether new factors have come into play. 

%\item
\textit{\textbf{\RQTwo}}
\git and social coding platforms offers advanced code-sharing facilities, yet at least one study reported very limited code integration between the mainline and the variant projects~\cite{businge:emse:2021}.
This research question investigates the overlap between the teams maintaining the mainline and variant forks, and how they communicate with one another.
As such we hope to identify impediments for co-evolution.
%\end{itemize}


The investigations are based on an online survey which we conducted with 112 maintainers involved in 105 active variant forks hosted on \gh.
%We performed an exploratory investigation of maintainers of variants hosted on \gh. We conducted an online survey that was answered by 112 maintainers of 105 active variant projects.
%
Our contributions are manifold:
we identify new reasons for creating and maintaining variant forks;
we identify and categorize different code reuse and change propagation practices between a variant and its mainline;
we confirm that little code integration occurs between a variant and its mainline, and uncover concrete reasons for this phenomenon;
and finally, we discuss the implications of these findings and how tools can help to achieve an efficient code integration and collaboration between mainlines and diverging variant forks.

%\sd{Need to squeeze this so it fits on a single page}

% \nd \textbf{We report a number of findings}: 1) Our study identifies a number of common fine-grained motivations for creating and maintaining variants that include: different goals\,/content\,/communities, customization, supporting personal projects, supporting the upstream, localization, up-taking a frozen feature.
% 2) Our study has identified a number of categories relating to different reuse practices of variants that include: integrating only bug\,/security fixes, updates only on specific features, all updates excluding specific features, and only updates that pass CI. For example, we have discovered software reuse of a family of applications from the ``cryptocurrency world'', which is dedicated project software ecosystem~\cite{tommens:2020}.
% 3) Our study extends the previous study by providing concrete reasons relating to reasons the little code integration observed between mainline and variants. For example, technically diverged variants, such as, those targeting different goals, those implementing different technologies, and those maintaining a specific feature of a mainline that was abandoned (a frozen feature). Based on these categories of variants, we have also discussed implications to tool building can help aid efficient code reuse between the mainlines and diverged variants.

