% !TEX root = 0main.tex

\section{Related Work}
%\section{Background and Related Work}
%no background is discussed
\label{sec:background}
Previous research has focused on (A) motivations for creating or maintaining variant forks; and (B) interaction between variant forks and their mainline.%, and (iii) general studies on variant forking.



\subsection{Motivations for creating or maintaining variant forks}
\label{sec:motivations}

Several studies have investigated motivations for creating and maintaining variant forks. However, most of these studies were carried out on \sourceforge, pre-dating the advent of social coding platforms like \gh~\cite{Linus:2012Perspectives,Gregorio:2012,Viseur:2012Forks,Linus:2013CodeForking,Laurent:2008,Linus:2011ToFork}. Several of those early studies report
perceived controversy
%controversial perceptions\cd{"controversial perceptions" is probably not what you were looking for, but rather something along the lines of "they are perceived as controversial".}
around variant forks~\cite{Chua:Forking:2017,Dixion:2009Forks,Ernst:2010,Linus:2011ToFork,Linus:2014Hackers,Raymond:Cathedral:2001,pete}.
Jiang et al.~\cite{Lo:2017} state that, although forking may have been controversial in the OSS community, it is now encouraged as a built-in feature on \gh. They further report that developers create \emph{social forks} of repositories to submit pull requests, fix bugs, and add new features. Zhou et al.~\cite{Zhou:2020} conclude that most variant forks started as social forks
and that perceptions of forks have changed with the advent of \gh.
Robles and Gonz{\'a}lez-Barahona~\cite{Gregorio:2012} carried out a comprehensive pre-\gh study on a carefully filtered list of 220 potential forks referenced on Wikipedia.
They report motivations and outcomes for forking on these 220 projects.
 %The authors assumed that a fork is significant if a reference to it appears in the English Wikipedia.\ad{@John: I missing the point of this last sentence. What are the findings of \cite{Gregorio:2012}?}

The literature has uncovered a number of motivations for creating variants. Below, we present those where both the mainline and variant co-evolve together. The motivation of \emph{reviving an abandoned project} is not considered in this study since it does involve co-evolution if the variants.

%\cd{Did you include all reasons for the mainline and variant to co-evolve? Then you could rephrase to avoid that your selection doesn't need to come across as subjective in ``some of the reported reason''.}
% \ad{@John: Where do the following reasons come from? Is this from \cite{Gregorio:2012} or from additional references (and in that case, which ones?)}

% \sd{@John: I would like to see a better motivation on where these 4 categories come from.
% Is this your own (ad hoc) classification?
% Does it stem from another paper?
% Is this a result of a (shallow) literature survey?
% Is this what came out of your analysis of the survey responses?}

\begin{list}{$\circ$}{}
   \item \textit{Technical (addition of functionality).} Sometimes developers want to include new functionality into the project, but the main developer(s) do not accept the contribution. An example is \textsf{Poppler}, a fork of \textsf{xpdf} relying on the \textsf{poppler} library~\cite{Gregorio:2012}.

    \item \textit{Governance disputes.} Some contributors from the community create a variant project because they feel that their feedback is not heard, or because the maintainers of the mainline are unresponsive or too slow at accepting their patches. A well-known example is a fork of \textsf{GNU Emacs} (originally \textsf{Lucid}) which was created as a result of the significant delays in bringing out a new version to support the Energize C++ IDE~\cite{Wheeler:2015Forking}.

\item \textit{Legal issues.} This includes disagreements on the license and trademarks, and changes to conform to rules and regulations. An example is \textsf{X.Org}, which originated from \textsf{XFree86} ~\cite{Gregorio:2012,Wheeler:2015Forking}. \textsf{XFree86} was originally \textsf{MIT/X} open source license that is GPL-compatible and then was changed to one that was not GPL-compatible. This caused many practical problems and a serious uproar in the community, resulting in  the project fork \textsf{X.Org}.

\item \textit{Personal reasons.} In some situations, the developer team disagrees on fundamental issues (beyond mere technical matters) related to the software development process and the project. An example is the \textsf{OpenBSD} fork from \textsf{NetBSD}. One of the developers of \textsf{NetBSD} had a disagreement with the rest of the core developers and decided fork and focus his efforts on \textsf{OpenBSD}~\cite{theo:openbsd}.
%\tm{So what was the fundamental issue of disagreement?}

%The variant was created because some of the contributors felt that their feedback was not heard or maintainers were accepting patches too slowly in the mainline project.

% \item \textit{Experimental}: Creating a fork to try out and test new features that will be integrated into the mainline. This category can be seen as a subcategory of technical reasons.
\end{list}

% \ad{@John: I tried to iterate on this section, but it's very difficult because it's pretty messy currently. Since I don't know the content of each paper, I don't know how to organise these references in a more structured and smooth way. Consider for example \cite{Zhou:2020} being used in 7 different places, with different ``reported findings''.}

Focusing on variant forks in the Android ecosystem, Businge et al.~\cite{businge:2018icsme} found that re-branding, simple customizations, feature extension, and implementation of different but related features are the main motivations to create forks of Android apps.
Zhou et al.~\cite{Zhou:2020} interviewed 18 developers of hard forks on \gh to understand reasons for forking in social coding environments that explicitly support forking. The motivations they observed align with the findings of the aforementioned studies.

Sung et al.~\cite{sung:ICSE:2020} investigated variant forks in an industrial case study to uncover the implications of frequent merges from the mainline and the resulting merge conflicts in the variant forks. They implemented a tool that can automatically resolve up to 40\% of 8 types of mainline-induced build breaks.

While the pre-\gh studies reported
perceived controversy
% controversial perceptions\cd{Again, I don't think it is the perceptions that are controversial themselves. Only that the forks are perceived as controversial. See above remark.}
around variant forks, Zhou et al.~\cite{Zhou:2020} report that this controversy has reduced with the advent of \gh. Jiang et al.~\cite{Lo:2017} report that, while forking is considered controversial in traditional OSS communities, it is actually embraced as a built-in feature in \gh.
%\cd{This sentence already featured in the beginning of the section.} The authors further report that developers fork repositories to submit pull requests, fix bugs, add new features and keep copies (social forks).
%Zhou et al.~\cite{Zhou:2020} also report that many variant forks actually start as social forks.
Our study builds on these previous studies to identify whether the motivations for variant forks are still the same or whether new factors have come into play.

\subsection{Interaction between variant forks and their mainline}
We have only encountered two studies that investigated the interaction between variant forks and mainlines~\cite{Zhou:2020,businge:emse:2021}.
%\az{I think we need to define the mainline here first. Maybe also define the used terms in a subsection "terminology". For example mainline} \jb{The only terminologies we have are variants and mainline. I plan to define them in the introduction section}
Zhou et al.~\cite{Zhou:2020} conducted 18 semi-structured developer interviews. Many respondents indicated being interested in coordination across repositories, either for eventually merging changes back into the mainline, or to monitor activity in the mainline repository and select and integrate interesting updates into their variant project.
%\tm{"to incorporate select or all changes"? Do not know what that means.}
Businge et al.~\cite{businge:emse:2021} also investigated the interaction between mainline and variants. The authors quantitatively investigated code propagation among variants and their mainline in three software ecosystems. They found that only about 11\% of the 10,979 mainline--variant pairs had integrated code between them.
Since the mainlines and variants share a common code base, and with the collaborative maintenance facilities of \git and the pull-based development model,
one would expect more interactions between the mainline and its variants. We hypothesise that there are some impediments to enable such interactions.
Since the two aforementioned studies do not report any such impediments,
we decided to carry an exploratory qualitative survey with variant maintainers to identify possible impediments.

%Like Zhou et al.~\cite{Zhou:2020}, in this study we carry out a qualitative study to find out if variant developers integrate code to and from the mainline repositories.
%\cd{Consider re-arranging so the reader does not read about zhou, followed by businge, followed by zhou.}
%\cd{I'm missing a conclusion about the related work that spoonfeeds reviewers the need for one more study and highlights the differences to this study. You could either summarise the differences at the end of this section, or sprinkle them throughout this section (e.g., "our work, in contrast, studies"). This is particularly important for the interview studies.}

%\subsection{General studies on variant forking}
% Like the between variant fork and mainline, we did not find many studies that relate to general studies on variant forking.
% One recent study that we came across investigated variant forks is that of Sung et al.~\cite{sung:ICSE:2020}. The authors conduct an industrial case study of the implications of frequent merges from mainline and the resulting merge conflicts in a variant fork. They implemented a tool that can automatically resolve up-to 40\% of the list of eight mainline induced build breaks. \az{the last two paper mentioned are already discussed above.}
